\documentclass{beamer}
\beamertemplatenavigationsymbolsempty
\usetheme{Goettingen}
\setbeamertemplate{footline}[frame number]

\mode<presentation>
\title{There Is No Largest Prime number}
\author[Euclid]{Euclid of Alexandria \\ \texttt{euclid@alexandria.edu}}
\institute{University of Alexandria}
\date[ISPN '80]{27th International Symposium of Prime Numbers}

\begin{document}
\begin{frame}
  \titlepage
\end{frame}

\AtBeginSection[]{ \begin{frame}\frametitle{Outline} \tableofcontents[currentsection]\end{frame} }

\section{Motivation}
\subsection{The Basic Problem That We Studied}

\begin{frame}
  \frametitle{What are Prime Numbers}
  \begin{definition}
    A \alert{prime number} is a number that has exactly two divisors.
  \end{definition}
  \pause
  \begin{example}
    \begin{itemize}
      \item<2-> 2 is prime (two divisors: 1 and 2).

      \item<3-> 3 is prime (two divisors: 1 and 3).

      \item<4-> 4 is not prime (\alert{three} divisors: 1, 2, and 4).
    \end{itemize}
  \end{example}
\end{frame}

\begin{frame}
  \frametitle{There Is No Largest Prime Number}
  \framesubtitle{The proof uses \textit{reductio ad absurdum}.}
  \begin{theorem}
    There is no largest prime number.
  \end{theorem}
  \begin{proof}
    \begin{enumerate}
      \item<1-> Suppose $p$ were the largest prime number.

      \item<2-> Let $q$ be the product of the first $p$ numbers.

      \item<3-> Then $q + 1$ is not divisible by any of them.

      \item<1-> But $q + 1$ is greater than $1$, thus divisible by some prime
            number not in the first $p$ numbers.\qedhere
    \end{enumerate}
  \end{proof}
  \uncover<4->{The proof used \textit{reductio ad absurdum}.}
\end{frame}

\begin{frame}
  \frametitle{What's Still To Do}
  \begin{columns}
    \column{.5\textwidth}
    \begin{block}{Answered Questions}
      How many primes are there?
    \end{block}
    \column{.5\textwidth}
    \begin{block}{Open Questions}
      Is every even number the sum of two primes?
    \end{block}
  \end{columns}
\end{frame}

\begin{frame}[fragile]
  \frametitle{An Algorithm For Finding Primes Numbers.} \begin{semiverbatim}
    \uncover<1->{\alert<0>{int main (void)}}
    \uncover<1->{\alert<0>{\{}}
    \uncover<1->{\alert<1>{  \alert<4>{std::}vector<bool> is_prime (100, true);}}
    \uncover<1->{\alert<1>{  for (int i = 2; i < 100; i++)}}
    \uncover<2->{\alert<2>{   if (is_prime[i])}}
    \uncover<2->{\alert<0>{   \{}}
    \uncover<3->{\alert<3>{     \alert<4>{std::}cout << i << " ";}}
    \uncover<3->{\alert<3>{     for (int j = i; j < 100;)}}
    \uncover<3->{\alert<3>{       is_prime[j] = false, j+=i);}}
    \uncover<2->{\alert<0>{   \}}}
    \uncover<1->{\alert<0>{  return 0;}}
    \uncover<1->{\alert<0>{\}}}
  \end{semiverbatim}
  \visible<4->{Note the use of \alert{\texttt{std::}}.}
\end{frame}

\end{document}



